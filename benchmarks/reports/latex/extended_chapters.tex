\documentclass{report}
\usepackage{amsmath, amssymb}
\usepackage{graphicx}
\usepackage{float}
\usepackage{tcolorbox}
\usepackage{xcolor}

\begin{document}


\chapter{Helmholtz Equation Benchmarks}

\section{Overview}

The Helmholtz equation $-u'' + k^2 u = f(x)$ on $[0,1]$ with boundary conditions $u(0) = \alpha$, $u(1) = \beta$ represents a fundamental model for wave propagation, quantum mechanics, and diffusion with reaction terms. We test five problems spanning low to high frequency content:

\begin{itemize}
\item \textbf{H1}: $k^2 = 1$ (low frequency, smooth baseline)
\item \textbf{H2}: $k^2 = 25$ (medium frequency)
\item \textbf{H3}: $k^2 = 100$ (high frequency, oscillatory)
\item \textbf{H4}: Exponential-oscillatory solution (tests rational representation of $e^{-2x}\sin(2\pi x)$)
\item \textbf{H5}: Polynomial solution (tests rational overhead without advantage)
\end{itemize}

\section{Results: Helmholtz $k^2=1$ (H1)}

\subsection{Convergence Plot}

Figure~\ref{fig:helmholtz_k1_conv} shows error vs DOF for all methods on the smooth sine problem $u(x) = \sin(\pi x)$.

\begin{figure}[H]
\centering
\includegraphics[width=0.9\textwidth]{figures/extended_benchmark/Helmholtz_Sin_k1_convergence.pdf}
\caption{Helmholtz $k^2=1$: Error vs DOF for all methods. Rational methods achieve exponential convergence (rate $\alpha \approx 21$) while polynomial methods show algebraic convergence.}
\label{fig:helmholtz_k1_conv}
\end{figure}

\subsection{Key Findings}

\begin{itemize}
\item \textbf{Rational [10/5]}: $1.84 \times 10^{-12}$ error (15 DOF) --- achieves \textbf{machine precision} with convergence rate $\alpha = 21.25$
\item \textbf{4th-order FD}: $8.81 \times 10^{-5}$ error (82 DOF) --- solid but requires $5\times$ more DOF
\item \textbf{Chebyshev}: $1.35 \times 10^{-3}$ error (33 DOF) --- good but not as efficient as rational
\item \textbf{2nd-order FD}: $7.16 \times 10^{-2}$ error (22-82 DOF) --- stagnates, inadequate baseline
\end{itemize}

\begin{figure}[H]
\centering
\includegraphics[width=0.9\textwidth]{figures/extended_benchmark/Helmholtz_Sin_k1_rates.pdf}
\caption{Convergence rates for Helmholtz $k^2=1$. Rational achieves exponential convergence while finite differences show expected algebraic rates.}
\label{fig:helmholtz_k1_rates}
\end{figure}

\section{Results: Helmholtz Exponential-Oscillatory (H4)}

This problem tests rational approximation of $u(x) = e^{-2x}\sin(2\pi x)$, which naturally involves both exponential decay and oscillation---ideal for rational representation.

\begin{figure}[H]
\centering
\includegraphics[width=0.9\textwidth]{figures/extended_benchmark/Helmholtz_Exp_k4_convergence.pdf}
\caption{Helmholtz exponential-oscillatory: Rationals excel at representing mixed exponential-trigonometric functions naturally.}
\label{fig:helmholtz_exp_k4_conv}
\end{figure}

\subsection{Key Findings}

\begin{itemize}
\item \textbf{Rational [10/5]}: $2.03 \times 10^{-7}$ error --- phenomenal accuracy
\item \textbf{Convergence rate}: $\alpha = 30.24$ (even better than pure sine!)
\item \textbf{Natural representation}: Rationals represent $e^{-2x}\sin(2\pi x)$ more naturally than polynomials
\item \textbf{4th-order FD}: Requires much higher resolution for comparable accuracy
\end{itemize}

\section{Summary}

On smooth Helmholtz problems, rational collocation demonstrates:
\begin{enumerate}
\item \textbf{Exponential convergence} (rates $\alpha = 19$--$30$)
\item \textbf{Unmatched accuracy} (machine precision with minimal DOF)
\item \textbf{Natural representation} of exponential-oscillatory solutions
\end{enumerate}

These results confirm that rationals excel in their intended domain: smooth problems with natural rational structure.


\chapter{Advection-Diffusion Benchmarks}

\section{Overview and Critical Importance}

The advection-diffusion equation
\[
-\varepsilon u'' + b u' = f(x)
\]
on $[0,1]$ models convection-dominated transport. The Péclet number $\text{Pe} = bL/\varepsilon$ determines problem character:
\begin{itemize}
\item $\text{Pe} < 1$: Diffusion-dominated, smooth
\item $1 < \text{Pe} < 10$: Balanced regime
\item $\text{Pe} > 10$: Convection-dominated, boundary layers
\item $\text{Pe} > 100$: Sharp boundary layers (extremely challenging)
\end{itemize}

\textbf{This chapter contains CRITICAL findings} demonstrating catastrophic failure of rational methods on boundary layer problems.

\section{Test Problems}

\begin{itemize}
\item \textbf{AD1}: $\varepsilon = 0.1$, $\text{Pe} = 10$ (mild layer)
\item \textbf{AD2}: $\varepsilon = 0.01$, $\text{Pe} = 100$ (noticeable layer)
\item \textbf{AD3}: $\varepsilon = 0.001$, $\text{Pe} = 1000$ (sharp layer) --- \textbf{CRITICAL TEST}
\item \textbf{AD4}: Interior layer (turning point)
\item \textbf{AD5}: Smooth advection-dominated (no layer)
\end{itemize}

\section{CRITICAL: Sharp Boundary Layer ($\varepsilon = 0.001$)}

\subsection{The Problem}

Exact solution:
\[
u(x) = \frac{e^{-x/\varepsilon} - e^{-1/\varepsilon}}{1 - e^{-1/\varepsilon}}
\]
exhibits exponentially thin boundary layer of thickness $O(\varepsilon) = O(0.001)$ at $x=0$.

\subsection{Catastrophic Rational Failure}

\begin{figure}[H]
\centering
\includegraphics[width=0.9\textwidth]{figures/extended_benchmark/AdvDiff_Sharp_eps0.001_convergence.pdf}
\caption{Sharp boundary layer ($\varepsilon=0.001$): Rational methods catastrophically fail with 64--76\% L$^2$ error, essentially unable to represent the sharp gradient. Convergence is nearly flat ($\alpha \approx 0.3$).}
\label{fig:advdiff_sharp_conv}
\end{figure}

\textbf{Results Summary:}

\begin{center}
\begin{tabular}{lccc}
\hline
\textbf{Method} & \textbf{DOF} & \textbf{L$^2$ Error} & \textbf{Status} \\
\hline
Rational [6/3] & 9 & \textcolor{red}{\textbf{76.0\%}} & \textcolor{red}{\textbf{FAIL}} \\
Rational [10/5] & 15 & \textcolor{red}{\textbf{64.6\%}} & \textcolor{red}{\textbf{FAIL}} \\
4th-order FD & 82 & 7.77\% & Marginal \\
Chebyshev N=32 & 33 & 98.7\% & Also struggles \\
\hline
\end{tabular}
\end{center}

\subsection{Critical Observations}

\begin{enumerate}
\item \textbf{Catastrophic failure}: 64--76\% L$^2$ error is \textbf{completely unacceptable}
\item \textbf{No improvement with refinement}: Increasing from [6/3] to [10/5] \textbf{barely helps} (76\% $\to$ 64.6\%)
\item \textbf{Convergence essentially stagnates}: $\alpha = 0.32$ (nearly zero convergence rate)
\item \textbf{Fundamental limitation}: Rationals \textbf{cannot represent sharp gradients} without introducing poles
\end{enumerate}

\begin{figure}[H]
\centering
\includegraphics[width=0.9\textwidth]{figures/extended_benchmark/AdvDiff_Sharp_eps0.001_rates.pdf}
\caption{Convergence rates for sharp boundary layer. Rational convergence rate $\alpha \approx 0.3$ indicates fundamental failure---the method is not converging.}
\label{fig:advdiff_sharp_rates}
\end{figure}

\section{Why Rationals Fail on Boundary Layers}

\subsection{Mathematical Explanation}

Rational functions $P(x)/Q(x)$ can represent:
\begin{itemize}
\item Smooth exponentials: $e^{ax}$ via Padé approximants
\item Smooth oscillations: $\sin(\omega x)$ via continued fractions
\item Algebraic singularities: $(x-a)^{-\alpha}$ naturally
\end{itemize}

But they \textbf{cannot efficiently represent} sharp, localized gradients like $e^{-x/\varepsilon}$ for $\varepsilon \ll 1$ because:
\begin{enumerate}
\item \textbf{Steep gradients require near-poles}: To approximate $e^{-x/0.001}$, the denominator $Q(x)$ would need zeros very close to $[0,1]$
\item \textbf{Numerical instability}: Near-poles cause catastrophic cancellation
\item \textbf{Global representation}: Rationals are \textbf{global} approximants; a boundary layer is \textbf{local}
\end{enumerate}

\subsection{Comparison with Other Methods}

\begin{itemize}
\item \textbf{Spectral methods}: Also struggle (Gibbs-like oscillations) but less catastrophically
\item \textbf{High-order FD}: Can resolve with sufficient refinement ($n \geq 80$ achieves 7.8\% error)
\item \textbf{Adaptive FD/FEM}: Can cluster points in boundary layer (not tested here)
\item \textbf{Exponential fitting}: Specialized methods designed for these problems
\end{itemize}

\section{Recommendations}

\subsection{When to AVOID Rationals}

\textbf{DO NOT use rational collocation when:}
\begin{enumerate}
\item $\varepsilon < 0.01$ in advection-diffusion problems
\item Péclet number $\text{Pe} > 100$
\item Boundary layer thickness $< 1\%$ of domain
\item Any problem with sharp, localized gradients
\end{enumerate}

\subsection{Alternative Methods for Boundary Layers}

\begin{enumerate}
\item \textbf{Adaptive mesh refinement} with standard FD/FEM
\item \textbf{Exponential fitting methods} (Il'in scheme, Scharfetter-Gummel)
\item \textbf{Upwind schemes} for convection-dominated problems
\item \textbf{Boundary layer resolving meshes} (Shishkin mesh)
\end{enumerate}

\section{Summary}

\textbf{Key Finding}: Rational collocation \textbf{catastrophically fails} on problems with boundary layers when $\varepsilon \leq 0.01$. This is a \textbf{fundamental limitation}, not a matter of insufficient resolution.

\begin{tcolorbox}[colback=red!10, colframe=red!50!black, title=Critical Warning]
\textbf{CRITICAL}: Before using rational collocation, verify that your problem does NOT contain:
\begin{itemize}
\item Boundary layers ($\varepsilon < 0.01$)
\item Sharp gradients (length scales $< 1\%$ of domain)
\item Discontinuities (see next chapter)
\end{itemize}
If present, use alternative methods. Rational collocation will produce \textbf{completely wrong results}.
\end{tcolorbox}


\chapter{Reaction-Diffusion Benchmarks}

\section{Overview}

The reaction-diffusion equation
\[
-u'' + c u = f(x)
\]
on $[0,1]$ models chemical reactions, population dynamics, and damped oscillations. For $c > 0$, solutions exhibit exponential boundary layers similar to advection-diffusion, but crucially these are \textbf{smooth} exponentials, not sharp gradients.

\section{Test Problems}

\begin{itemize}
\item \textbf{RD1}: $c = 1$ (mild reaction, smooth oscillatory)
\item \textbf{RD2}: $c = 10$ (exponential boundary layers, $u = \sinh(\sqrt{c}\,x)/\sinh(\sqrt{c})$)
\item \textbf{RD3}: $c = 100$ (steep exponential, $u = \sinh(10x)/\sinh(10)$)
\item \textbf{RD4}: Damped oscillatory ($u = e^{-x}\cos(5\pi x)$)
\item \textbf{RD5}: Polynomial (no rational advantage)
\end{itemize}

\section{Results: Exponential Layers ($c = 10$)}

\subsection{Rationals Excel on Smooth Exponentials}

\begin{figure}[H]
\centering
\includegraphics[width=0.9\textwidth]{figures/extended_benchmark/ReacDiff_Exp_c10_convergence.pdf}
\caption{Reaction-diffusion with $c=10$: Rationals achieve spectacular accuracy on smooth exponential $\sinh(\sqrt{10}\,x)$ because the boundary layers are smooth, unlike the sharp discontinuous layers in advection-diffusion.}
\label{fig:reacdiff_exp_conv}
\end{figure}

\textbf{Results Summary:}

\begin{center}
\begin{tabular}{lccc}
\hline
\textbf{Method} & \textbf{DOF} & \textbf{L$^2$ Error} & \textbf{Conv. Rate} \\
\hline
\textbf{Rational [10/5]} & 15 & $\mathbf{4.52 \times 10^{-13}}$ & $\alpha = 19.75$ \\
Chebyshev N=32 & 33 & $4.51 \times 10^{-4}$ & $\alpha = 2.08$ \\
4th-order FD (n=80) & 82 & $5.45 \times 10^{-5}$ & $\alpha = 1.97$ \\
2nd-order FD (n=80) & 82 & $2.32 \times 10^{-1}$ & $\alpha \approx 0$ \\
\hline
\end{tabular}
\end{center}

\subsection{Key Findings}

\begin{enumerate}
\item \textbf{Machine precision}: Rational [10/5] achieves $4.52 \times 10^{-13}$ error
\item \textbf{Exponential convergence}: $\alpha = 19.75$ (spectral convergence)
\item \textbf{Natural representation}: $\sinh(x)$ is naturally represented as a ratio of exponentials
\item \textbf{Factor of $10^9$ better}: Rational error is \textbf{one billion times smaller} than 4th-order FD
\end{enumerate}

\section{Results: Polynomial Solution ($c = 4$)}

This problem tests rational overhead when the solution is polynomial: $u(x) = x^2(1-x)^2$.

\begin{figure}[H]
\centering
\includegraphics[width=0.9\textwidth]{figures/extended_benchmark/ReacDiff_Polynomial_c4_convergence.pdf}
\caption{Polynomial solution: Rationals still achieve machine precision but offer no special advantage---spectral methods would be equally efficient.}
\label{fig:reacdiff_poly_conv}
\end{figure}

\textbf{Results Summary:}

\begin{center}
\begin{tabular}{lccc}
\hline
\textbf{Method} & \textbf{DOF} & \textbf{L$^2$ Error} & \textbf{Notes} \\
\hline
Rational [6/3] & 9 & $4.23 \times 10^{-17}$ & Machine precision \\
Chebyshev N=32 & 33 & $1.21 \times 10^{-4}$ & Good \\
4th-order FD (n=80) & 82 & $1.35 \times 10^{-5}$ & Solid \\
\hline
\end{tabular}
\end{center}

\subsection{Observations}

For polynomial solutions:
\begin{itemize}
\item Rationals still achieve machine precision (overhead of rational representation is low)
\item No advantage over spectral methods (both would achieve $\sim 10^{-16}$ error)
\item 4th-order FD is competitive and much faster (solve time: 0.87ms vs 24.83ms)
\end{itemize}

\section{Why Rationals Excel Here (vs Advection-Diffusion)}

\subsection{Smooth vs Sharp Gradients}

\textbf{Reaction-Diffusion ($c=10$):}
\[
u(x) = \frac{\sinh(\sqrt{10}\,x)}{\sinh(\sqrt{10})} = \frac{e^{\sqrt{10}\,x} - e^{-\sqrt{10}\,x}}{e^{\sqrt{10}} - e^{-\sqrt{10}}}
\]
\begin{itemize}
\item \textbf{Smooth everywhere}: All derivatives are bounded
\item \textbf{Natural rational form}: Already expressed as ratio of exponentials
\item \textbf{Moderate gradients}: $\sqrt{10} \approx 3.16$ is manageable
\end{itemize}

\textbf{Advection-Diffusion ($\varepsilon=0.001$):}
\[
u(x) = \frac{e^{-x/0.001} - e^{-1/0.001}}{1 - e^{-1/0.001}}
\]
\begin{itemize}
\item \textbf{Nearly discontinuous}: $e^{-x/0.001}$ drops from 1 to $\sim 0$ in distance $0.001$
\item \textbf{Extreme gradients}: $|u'(0)| \sim 1000$ (derivative is huge)
\item \textbf{Requires near-poles}: Would need $Q(x)$ with zeros $\sim 0.001$ from boundary
\end{itemize}

\section{Summary}

Reaction-diffusion benchmarks demonstrate that rationals:
\begin{enumerate}
\item \textbf{Excel on smooth exponentials} (convergence rate $\alpha \approx 20$)
\item \textbf{Achieve machine precision} with minimal DOF
\item \textbf{Naturally represent} $\sinh$, $\cosh$, and exponential-oscillatory functions
\item \textbf{Still work on polynomials} (though no special advantage)
\end{enumerate}

The key distinction from advection-diffusion: \textbf{smoothness}. As long as the solution and its derivatives are bounded, rationals excel. Sharp gradients and near-discontinuities cause catastrophic failure.


\chapter{Variable Coefficient Benchmarks}

\section{Overview}

The variable coefficient problem
\[
-(a(x) u')' = f(x)
\]
on $[0,1]$ models diffusion with spatially varying conductivity. The coefficient $a(x)$ can be smooth, oscillatory, discontinuous, or singular, each presenting unique challenges.

\section{Test Problems}

\begin{itemize}
\item \textbf{VC1}: $a(x) = 1 + x^2$ (smooth polynomial)
\item \textbf{VC2}: $a(x) = e^x$ (smooth exponential)
\item \textbf{VC3}: $a(x) = \begin{cases} 1 & x < 0.5 \\ 10 & x \geq 0.5 \end{cases}$ (discontinuous) --- \textbf{CRITICAL TEST}
\item \textbf{VC4}: $a(x) = 2 + \sin(20\pi x)$ (rapidly oscillating)
\item \textbf{VC5}: $a(x) = \sqrt{x + 0.01}$ (near-singular at $x=0$)
\end{itemize}

\section{CRITICAL: Discontinuous Coefficient}

\subsection{The Problem}

Coefficient has jump discontinuity:
\[
a(x) = \begin{cases}
1 & \text{if } x < 0.5 \\
10 & \text{if } x \geq 0.5
\end{cases}
\]

Solution is constructed to have \textbf{continuous $u$ and $u'$}, but $a(x)u'$ has a jump at $x=0.5$, meaning $u''$ is discontinuous.

\subsection{Catastrophic Rational Failure}

\begin{figure}[H]
\centering
\includegraphics[width=0.9\textwidth]{figures/extended_benchmark/VarCoeff_Discontinuous_convergence.pdf}
\caption{Discontinuous coefficient: Rational methods completely fail to represent discontinuities, achieving only 39--40\% L$^2$ error. Convergence rate $\alpha \approx 0.06$ indicates complete stagnation.}
\label{fig:varcoeff_disc_conv}
\end{figure}

\textbf{Results Summary:}

\begin{center}
\begin{tabular}{lccc}
\hline
\textbf{Method} & \textbf{DOF} & \textbf{L$^2$ Error} & \textbf{Conv. Rate} \\
\hline
Rational [6/3] & 9 & \textcolor{red}{\textbf{40.2\%}} & $\alpha = 0.06$ \\
Rational [10/5] & 15 & \textcolor{red}{\textbf{39.0\%}} & \textcolor{red}{\textbf{FAIL}} \\
2nd-order FD (n=80) & 82 & 125\% & Also struggles \\
\hline
\end{tabular}
\end{center}

\subsection{Critical Observations}

\begin{enumerate}
\item \textbf{Complete failure}: 39--40\% error is catastrophic
\item \textbf{No convergence}: $\alpha = 0.06 \approx 0$ (essentially flat)
\item \textbf{Refinement useless}: [6/3] vs [10/5] makes no meaningful difference
\item \textbf{Gibbs-like phenomena}: Rationals exhibit oscillations near discontinuity
\end{enumerate}

\begin{figure}[H]
\centering
\includegraphics[width=0.9\textwidth]{figures/extended_benchmark/critical_failures.pdf}
\caption{Critical failure comparison: Both boundary layer and discontinuous coefficient problems show catastrophic rational failure ($>10\%$ error threshold).}
\label{fig:critical_failures}
\end{figure}

\section{Why Rationals Fail on Discontinuities}

\subsection{Mathematical Explanation}

Rational functions $P(x)/Q(x)$ are:
\begin{itemize}
\item \textbf{Infinitely smooth} everywhere except at poles
\item \textbf{Cannot represent jumps} without introducing poles \emph{at} the discontinuity
\item \textbf{Globally coupled}: A discontinuity at $x=0.5$ affects approximation everywhere
\end{itemize}

This is analogous to \textbf{Gibbs phenomenon} in Fourier series:
\begin{itemize}
\item Global approximants (Fourier, polynomials, rationals) struggle with discontinuities
\item Local approximants (splines, finite elements) handle discontinuities naturally
\end{itemize}

\subsection{Comparison with Other Methods}

\begin{itemize}
\item \textbf{2nd-order FD}: Also struggles (125\% error) but can improve with refinement
\item \textbf{Finite Elements}: Would handle naturally via element interfaces
\item \textbf{DG methods}: Designed for discontinuities via upwind fluxes
\item \textbf{Adaptive refinement}: Can cluster resolution at $x=0.5$
\end{itemize}

\section{Recommendations}

\subsection{When to AVOID Rationals}

\textbf{DO NOT use rational collocation when:}
\begin{enumerate}
\item Coefficients $a(x)$, $b(x)$, or $c(x)$ are discontinuous
\item Solution or derivatives have jumps
\item Problem has interfaces, shocks, or contact discontinuities
\item Material properties change abruptly
\end{enumerate}

\subsection{Alternative Methods for Discontinuities}

\begin{enumerate}
\item \textbf{Finite Elements} with interface-aligned meshes
\item \textbf{Discontinuous Galerkin (DG)} methods
\item \textbf{Immersed interface methods}
\item \textbf{Extended FEM (XFEM)} for moving interfaces
\end{enumerate}

\section{Smooth Variable Coefficients}

For completeness, when $a(x)$ is smooth (polynomial, exponential, oscillatory), rational methods can work but offer no particular advantage over spectral methods. The polynomial and exponential cases showed that rationals still struggle compared to simpler alternatives.

\section{Summary}

\textbf{Key Finding}: Rational collocation \textbf{catastrophically fails} on problems with discontinuous coefficients. Convergence rate $\alpha \approx 0$ indicates the method is fundamentally unable to represent discontinuities.

\begin{tcolorbox}[colback=red!10, colframe=red!50!black, title=Critical Warning]
\textbf{CRITICAL}: If your problem has discontinuous:
\begin{itemize}
\item Coefficients ($a(x)$, $b(x)$, $c(x)$ with jumps)
\item Solutions (shocks, interfaces)
\item Derivatives (kinks, corners)
\end{itemize}
then rational collocation will produce \textbf{completely wrong results} (error $> 30\%$). Use finite elements or DG methods instead.
\end{tcolorbox}


\chapter{Cross-Method Analysis and Recommendations}

\section{Overall Method Performance Summary}

After testing 6 methods across 20+ problems spanning 4 differential operator types, we can now provide comprehensive guidance on method selection.

\section{Convergence Rate Summary}

Table~\ref{tab:convergence_summary} summarizes observed convergence rates $\alpha$ where error $\sim \text{DOF}^{-\alpha}$.

\begin{table}[H]
\centering
\caption{Convergence rates across all methods and problem types}
\label{tab:convergence_summary}
\begin{tabular}{lcccccc}
\hline
\textbf{Problem Type} & \textbf{2nd FD} & \textbf{4th FD} & \textbf{6th FD} & \textbf{Cheby} & \textbf{Leg} & \textbf{Rational} \\
\hline
Helmholtz (smooth) & 0.00 & 1.79 & --- & 2.08 & 2.04 & \textbf{21.25} \\
Helmholtz (exp-osc) & 0.00 & 1.88 & --- & 2.08 & 2.04 & \textbf{30.24} \\
ReacDiff (exp) & 0.00 & 1.97 & --- & 2.08 & 2.04 & \textbf{19.75} \\
ReacDiff (poly) & 0.01 & 2.34 & --- & 2.08 & 2.04 & --- \\
AdvDiff (mild) & 0.00 & 1.77 & --- & 0.00 & 0.00 & 4.43 \\
AdvDiff (sharp) & 0.00 & 2.25 & --- & 0.76 & 1.12 & \textcolor{red}{\textbf{0.32}} \\
VarCoeff (disc) & 0.00 & --- & --- & --- & --- & \textcolor{red}{\textbf{0.06}} \\
\hline
\end{tabular}
\end{table}

\subsection{Key Observations}

\begin{enumerate}
\item \textbf{Rationals: Exponential on smooth, zero on non-smooth}
   \begin{itemize}
   \item Smooth Helmholtz/ReacDiff: $\alpha = 19$--$30$ (spectacular)
   \item Boundary layer: $\alpha = 0.32$ (catastrophic)
   \item Discontinuous: $\alpha = 0.06$ (complete failure)
   \end{itemize}

\item \textbf{4th-order FD: Consistent algebraic convergence}
   \begin{itemize}
   \item Most problems: $\alpha \approx 1.8$--$2.3$ (close to theoretical 4)
   \item Never fails catastrophically
   \item Reliable general-purpose method
   \end{itemize}

\item \textbf{Spectral (Cheby/Leg): Smooth problem specialists}
   \begin{itemize}
   \item Smooth problems: $\alpha \approx 2$ (limited by few test points)
   \item Boundary layers: $\alpha \approx 0.8$--$1.1$ (struggles but better than rationals)
   \item Would show exponential convergence with more points
   \end{itemize}

\item \textbf{2nd-order FD: Inadequate baseline}
   \begin{itemize}
   \item $\alpha \approx 0$ (no convergence in tested range)
   \item Confirms our original critique: comparing rationals against 2nd-order FD is misleading
   \end{itemize}
\end{enumerate}

\section{Accuracy vs Cost Tradeoffs}

\begin{figure}[H]
\centering
\includegraphics[width=0.9\textwidth]{figures/extended_benchmark/Helmholtz_Sin_k1_convergence.pdf}
\caption{Typical smooth problem (Helmholtz $k^2=1$): Rationals achieve machine precision fastest, but spectral/high-order FD are competitive and more robust.}
\label{fig:accuracy_cost}
\end{figure}

\subsection{For 10$^{-6}$ Accuracy on Smooth Problems}

\begin{center}
\begin{tabular}{lcc}
\hline
\textbf{Method} & \textbf{DOF Required} & \textbf{Solve Time} \\
\hline
Rational [6/3] & $\sim 10$ & $\sim 80$ ms \\
Chebyshev & $\sim 20$ & $\sim 0.5$ ms \\
4th-order FD & $\sim 40$ & $\sim 1$ ms \\
2nd-order FD & $> 1000$ & $\sim 10$ ms \\
\hline
\end{tabular}
\end{center}

\textbf{Key Insight}: Rationals require fewest DOF but are slowest per solve (nonlinear optimization). For many applications, Chebyshev with $4\times$ more DOF is \textbf{faster overall} ($0.5$ ms vs $80$ ms).

\section{Method Selection Decision Tree}

\begin{tcolorbox}[colback=blue!5, colframe=blue!50!black, title=Method Selection Guide]

\textbf{Start here}:

\begin{enumerate}
\item \textbf{Does problem have any of these?}
   \begin{itemize}
   \item Boundary layers ($\varepsilon < 0.01$, Pe $> 100$)
   \item Discontinuous coefficients or solutions
   \item Sharp gradients (length scales $< 1\%$ domain)
   \end{itemize}
   \textbf{YES} $\Rightarrow$ \textcolor{red}{\textbf{DO NOT use rationals}}. Use FEM, DG, or adaptive FD.

   \textbf{NO} $\Rightarrow$ Continue to 2.

\item \textbf{Is solution smooth everywhere?}

   \textbf{YES} $\Rightarrow$ Continue to 3.

   \textbf{NO} $\Rightarrow$ Use finite elements or finite differences.

\item \textbf{Does solution involve exponentials or rational structure?}
   \begin{itemize}
   \item $e^{\pm ax}$, $\sinh(ax)$, $\cosh(ax)$
   \item $e^{ax}\sin(\omega x)$, $e^{ax}\cos(\omega x)$
   \item Known to be naturally expressible as $P(x)/Q(x)$
   \end{itemize}
   \textbf{YES} $\Rightarrow$ \textcolor{green}{\textbf{Rationals recommended}} (will achieve machine precision)

   \textbf{NO} $\Rightarrow$ Continue to 4.

\item \textbf{Is extreme accuracy needed (error $< 10^{-10}$)?}

   \textbf{YES} $\Rightarrow$ Use spectral methods (Chebyshev/Legendre) or rationals

   \textbf{NO} $\Rightarrow$ Use 4th-order FD (reliable, fast, robust)
\end{enumerate}

\end{tcolorbox}

\section{Detailed Recommendations by Problem Class}

\subsection{Poisson / Helmholtz (smooth forcing)}

\textbf{Best choice}: Spectral (Chebyshev/Legendre) or Rational

\begin{itemize}
\item Both achieve exponential convergence
\item Spectral is faster (linear solve vs nonlinear optimization)
\item Rational achieves slightly better accuracy (machine precision with fewer DOF)
\item 4th-order FD is solid backup (robust, predictable)
\end{itemize}

\subsection{Reaction-Diffusion (smooth exponentials)}

\textbf{Best choice}: Rational collocation

\begin{itemize}
\item Natural representation of $\sinh$, $\cosh$, exponential solutions
\item Achieves machine precision ($\sim 10^{-13}$)
\item Convergence rates $\alpha \approx 20$
\item Worth the computational cost for natural problem structure
\end{itemize}

\subsection{Advection-Diffusion}

\textbf{Depends critically on $\varepsilon$}:

\begin{itemize}
\item $\varepsilon \geq 0.1$: Rationals may work
\item $0.01 < \varepsilon < 0.1$: Use 4th-order FD or spectral
\item $\varepsilon \leq 0.01$: \textcolor{red}{\textbf{DO NOT use rationals}} (catastrophic failure)
   \begin{itemize}
   \item Use upwind FD, exponential fitting, or adaptive mesh
   \end{itemize}
\end{itemize}

\subsection{Variable Coefficients}

\textbf{Depends on smoothness of $a(x)$}:

\begin{itemize}
\item Smooth $a(x)$ (polynomial, exponential, oscillatory): 4th-order FD or spectral
\item Discontinuous $a(x)$: \textcolor{red}{\textbf{DO NOT use rationals}}
   \begin{itemize}
   \item Use finite elements with interface-aligned mesh
   \item Or discontinuous Galerkin (DG) methods
   \end{itemize}
\end{itemize}

\section{Cost-Benefit Analysis}

\subsection{Rational Collocation}

\textbf{Benefits}:
\begin{itemize}
\item Exponential convergence on smooth problems ($\alpha = 20$--$30$)
\item Machine precision with minimal DOF
\item Natural representation of exponential/rational structure
\end{itemize}

\textbf{Costs}:
\begin{itemize}
\item $50$--$1000\times$ slower than linear solvers (nonlinear optimization)
\item Catastrophic failure on non-smooth problems
\item Requires careful problem diagnosis
\item Implementation complexity (Bernstein basis, cleared form, constraints)
\end{itemize}

\textbf{Verdict}: Use only when (1) problem is verified smooth, (2) extreme accuracy needed, (3) solution has natural rational structure. Otherwise, spectral or 4th-order FD are safer.

\subsection{Spectral Methods (Chebyshev/Legendre)}

\textbf{Benefits}:
\begin{itemize}
\item Exponential convergence on smooth problems
\item Fast linear solves
\item Well-established theory and implementations
\item Robust on smooth problems
\end{itemize}

\textbf{Costs}:
\begin{itemize}
\item Gibbs oscillations near discontinuities
\item Dense matrices (not suitable for very large problems)
\item Require smooth forcing and coefficients
\end{itemize}

\textbf{Verdict}: Excellent general-purpose method for smooth problems. First choice unless problem has natural rational structure.

\subsection{4th-Order Finite Differences}

\textbf{Benefits}:
\begin{itemize}
\item Reliable algebraic convergence ($\alpha \approx 2$)
\item Never fails catastrophically
\item Fast linear solves
\item Simple to implement
\item Sparse matrices (scalable)
\end{itemize}

\textbf{Costs}:
\begin{itemize}
\item Cannot achieve machine precision (limited by algebraic convergence)
\item Requires more DOF than spectral/rational for high accuracy
\end{itemize}

\textbf{Verdict}: Best general-purpose method. Use as default unless extreme accuracy ($< 10^{-10}$) is needed.

\section{Revisiting the Original Claims}

\subsection{The "2,000,000$\times$ Improvement" Claim}

Our original comparison showed rational collocation achieving $\sim 10^{-12}$ error while 2nd-order FD achieved $\sim 10^{-3}$ error, suggesting $10^9$ improvement.

\textbf{Reality check with proper baselines}:
\begin{itemize}
\item \textbf{vs 2nd-order FD}: Yes, $\sim 10^6$--$10^9\times$ better (but 2nd-order is obsolete)
\item \textbf{vs 4th-order FD}: $\sim 10^3$--$10^6\times$ better on smooth problems
\item \textbf{vs Spectral methods}: $\sim 10$--$100\times$ better on smooth problems
\item \textbf{On boundary layers}: Rationals are $\infty\times$ \textbf{WORSE} (complete failure)
\end{itemize}

\subsection{Honest Assessment}

Rational collocation is:
\begin{enumerate}
\item \textbf{Excellent} for smooth problems with exponential/rational structure
\item \textbf{Competitive} with spectral methods on general smooth problems
\item \textbf{Completely unsuitable} for non-smooth problems
\item \textbf{Not a replacement} for general-purpose solvers
\end{enumerate}

The method fills a \textbf{niche}: problems that are smooth, have natural rational structure, and require extreme accuracy.

\section{Summary}

The extended benchmark suite demonstrates that:

\begin{enumerate}
\item \textbf{Context matters}: No single method is universally best
\item \textbf{Rational collocation has a niche}: Smooth exponential/rational problems
\item \textbf{Critical limitations exist}: Catastrophic failure on boundary layers and discontinuities
\item \textbf{Proper baselines essential}: Comparing against 2nd-order FD was misleading
\item \textbf{Robust methods preferred}: 4th-order FD is often the best practical choice
\end{enumerate}

Users must carefully diagnose their problem characteristics before selecting a method. The decision tree in this chapter provides guidance for making this choice.


\end{document}
